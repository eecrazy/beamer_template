\documentclass[11pt,UTF8]{ctexbeamer}
\mode<article> % 仅应用于article版本
{
  \usepackage{beamerbasearticle}
  \usepackage{fullpage}
  \usepackage{hyperref}
}
%% 下面的包控制beamer的风格,可以根据自己的爱好修改
\usepackage{beamerthemesplit}   % 使用split风格
\usepackage{beamerthemeshadow}  % 使用shadow风格
%% 这些包是可能会用到的,不必修改
\usepackage{pgf,pgfarrows,pgfnodes,pgfautomata,pgfheaps}
\usepackage{amsmath,amssymb}
\usepackage{graphics}
\usepackage{multimedia}
%% 下面的代码用来读入Logo图象
\pgfdeclaremask{logomask}{pku-tower-mask}
\pgfdeclareimage[mask=logomask,height=1.5cm]{logo}{pku-tower}
\pgfdeclaremask{beidamask}{beida-mark-mask}
\pgfdeclareimage[mask=beidamask,height=0.25cm]{beida}{beida-mark}
\pgfdeclaremask{titlemask}{pku-lake2-mask}
\pgfdeclareimage[mask=titlemask,height=2.5cm]{title}{pku-lake2}

% \logo{\vbox{\hbox{\hfill\pgfuseimage{logo}}}} 
%设置logo图标
%% 定义一些自选的模板,包括背景、图标、导航条和页脚等,修改要慎重
\beamertemplateshadingbackground{red!0}{structure!10}
\beamertemplatesolidbackgroundcolor{white!90!blue}
\beamertemplatetransparentcovereddynamic
\beamertemplateballitem
\beamertemplatenumberedballsectiontoc
% \beamertemplatelargetitlepage
\beamertemplateboldpartpage
\makeatletter
\usefoottemplate{ %重新定义页脚,加入作者,单位,单位图标,和文档标题
  \vbox{\tiny%
    \hbox{%
      \setbox\beamer@linebox=\hbox to\paperwidth{%
        \hbox to.5\paperwidth{\hfill\tiny\color{white}\textbf{\insertshortauthor\quad\insertshortinstitute}\hskip.1cm\lower 0.35em\hbox{}\hskip.3cm}%
        \hbox to.5\paperwidth{\hskip.3cm\tiny\color{white}\textbf{\insertshorttitle}\hfill}\hfill}%
      \ht\beamer@linebox=2.625ex%
      \dp\beamer@linebox=0pt%
      \setbox\beamer@linebox=\vbox{\box\beamer@linebox\vskip1.125ex}%
      \color{structure}\hskip-\Gm@lmargin\vrule width.5\paperwidth
      height\ht\beamer@linebox\color{structure!70}\vrule width.5\paperwidth
      height\ht\beamer@linebox\hskip-\paperwidth% 
      \hbox{\box\beamer@linebox\hfill}\hfill\hskip-\Gm@rmargin}
  }
}
\makeatother
%% 填写作者,单位,日期,标题等文档信息
\title{Weekly Report}
\subtitle{}
\author[Li Zhongyang]{Li Zhongyang}
\institute[ HIT-SCIR]{
  Research Center for Social Computing and Information Retrieval\\
  Department of Computer Science and Technology\\
  Harbin Institute of Technology
}
\date[]{\today}
\subject{Report}
% \titlegraphic{\pgfuseimage{title}}
% 在每个Section前都会加入的Frame:outline
\AtBeginSection[]{ 
  \frame<handout:0>{
    \frametitle{Outline}
    \tableofcontents[current,currentsubsection]
  }
}
\begin{document}
\frame[plain]{\titlepage} % 产生主题页,plain选项表示不显示页眉页脚等内容
% \section<presentation>*{Outline}   % 带*号表示不加到目录中
% \frame{
%   \nameslide{outline}
%   \frametitle{Outline}
%   \tableofcontents[pausesections]
%   \note{At most 1 minute for the outline}
% }

\part{Weekly Report}
\frame{\partpage}

\section{Done In the Last Week}

\frame[<+-| alert@+>]{
  \frametitle{Read Papers}
  \begin{itemize}
  \item Neural Networks
    \begin{enumerate}
    \item Memory networks
    \item Seq2Seq model 
    \item Attention mechanism
    \end{enumerate}
  \item Causal and temporal
    \begin{enumerate}
    \item How to judge causal relationship between events
    \item How to judge temporal relationship between events
    \end{enumerate}
  \end{itemize}
}

\frame[<+-| alert@+>]{
  \frametitle{Experiments}
  \begin{itemize}
  \item Data Processing
    \begin{enumerate}
    \item extract 30,000 causal instances
    \item each contains ‘因为,所以’
    \end{enumerate}
  \item The Seq2Seq Model with Adversarial Training
    \begin{enumerate}
    \item Seq2Seq model 
    \item Adversarial Training
    \end{enumerate}
  \end{itemize}
}

\section{To Do Next Week}

\frame[<+-| alert@+>]{
  \frametitle{\secname}
  \begin{enumerate}
  \item Read more papers
  \item Continue the experiment
  \end{enumerate}
}

\part{Slides Tools}
\frame{\partpage}


\section{Tools Like Powerpoint}

\frame[<+->]{
  \frametitle{Tools Like Powerpoint}
  \begin{itemize}
  \item Advantage
    \begin{enumerate}
    \item What you see is what you get
    \item All done in one software
    \item Easy to learn
    \end{enumerate}
  \item Disadvantage
    \begin{enumerate}
    \item Commonly the software is not free
    \item It depand on the software
    \item It's hard for much formula
    \end{enumerate}
  \end{itemize}
}

\section{\TeX\ Tools}
\frame[<+-| alert@+>]{
  \frametitle{\TeX\ Tools}
  \begin{enumerate}
  \item Base PDF file
  \item Deal with mathematic formula easily
  \item Professional typeset 
  \item Plain text, easy to reuse
  \end{enumerate}
}
\frame[<+->]{
  \frametitle{\TeX\ Slide Tools}
  \begin{description}
  \item[Beamer] A~standard~\LaTeX\ ~Document~ class,\\
    Need~ no other post progress program\\
    Work with other \LaTeX\ packages smoothly
  \item[foiltex] Work with most of the~available~\LaTeX\ ~commands~and~environments\\
    Use Macro $\backslash$Mylogo put some graphic as the logo
  \item[prosper] Automatically generated table of contents, Portrait slides support \\
    and possible to include notes in your presentation
  \item[pdfscreen] Create document both fit to read in computer and for print
  \item[seminar] A simple \LaTeX style designed for seminar presentations.
  \item[TeXPower] A \LaTeX\ style \textsl{texpower.sty}
  \end{description}
}


% \section*{Reference}
% \frame{
%   \frametitle{Reference}
% }
\frame{
    \begin{centering}
      \Huge Thank You! \\
      Questions and Suggestions?\par
    \end{centering}
}
\end{document}
